%%=============================================================================
%% Conclusie
%%=============================================================================

\chapter{Conclusie}
\label{ch:conclusie}

% TODO: Trek een duidelijke conclusie, in de vorm van een antwoord op de
% onderzoeksvra(a)g(en). Wat was jouw bijdrage aan het onderzoeksdomein en
% hoe biedt dit meerwaarde aan het vakgebied/doelgroep? 
% Reflecteer kritisch over het resultaat. In Engelse teksten wordt deze sectie
% ``Discussion'' genoemd. Had je deze uitkomst verwacht? Zijn er zaken die nog
% niet duidelijk zijn?
% Heeft het onderzoek geleid tot nieuwe vragen die uitnodigen tot verder 
%onderzoek?

Dit onderzoek heeft aangetoond dat het vinden van het gepaste AI-framework voor een wayfinding context niet vanzelfsprekend is, er is namelijk weinig onderzoek gedaan omtrent de samenwerking van AI en AR-frameworks. De 'jonge' leeftijd van deze technologieën speelt hier parte, zoals eerder werd vermeld werden deze pas in 2017 vrijgegeven. Dit heeft als gevolg dat complexere uitwerkingen nog niet hun optimum hebben bereikt.

Om de AI-kant van dit onderzoek te evalueren werden verschillende tests uitgevoerd op frameworks die werden geselecteerd volgens verscheidene criteria, deze waren gericht op de verschillende AI-eigenschappen die meer info gaven over de correctheid van elk framework. Om deze frameworks zo optimaal mogelijk te testen werden de proeven op verschillende plaatsen uitgevoerd, ook werd er rekening gehouden met consistentie. De consistentie werd beproeft door elke  test op een iteratieve wijze uit te voeren. Het aantal iteraties werd ook zorgvuldig gekozen, door een oneven getal te nemen werden ex aequo's vermeden.

De resultaten van het onderzoek stemden niet overeen met de verwachtingen uit het literatuuronderzoek. TensorFlow heeft een goede reputatie binnen de AI-wereld, maar in dit onderzoek voldeed het niet aan zijn normen. CoreML scoorde dan wel beter als verwacht, in sommige testen kon het framework een voorwerp met bijna 100 \% zekerheid aanduiden.
Uit de resultaten van deze tests was er al snel een verband waarneembaar , telkens scoorde CoreML beter dan TensorFlow Lite. 

%Om de werking tussen ARKit en het CoreML-framework verder te onderzoeken werd er ook een minimalistische uitwerken gerealiseerd met Xcode. Uit deze proof of concept kan men concluderen dat het CoreML-framework in staat is om een op goede manier te communiceren met de visuele kant van de applicatie. Men is er zich van bewust dat deze proof of concept niet in direct verband staat met de wayfinding applicatie, deze uitwerking zou kunnen geoptimaliseerd worden door een applicatie te realiseren die hier meer mee in verband staat.

Uit dit onderzoek kan er dus geconcludeerd worden dat CoreML een geschikt AI-framework is voor de uitwerking van een wayfinding applicatie. Het beschikt over alle eigenschappen die noodzakelijk zijn om de weg op een correcte manier aan te geven. Er kan niet besloten worden dat CoreML het meest geschikte framework is voor 'In The Pocket', om deze conclusie te maken zouden alle opgegeven frameworks uit de literatuurstudie moeten getest en geëvalueerd worden.

Dit onderzoek kan dus zeker nog verder uitgebreid worden, ten eerste kunnen alle frameworks uit de literatuurstudie getest en geëvalueerd worden volgens de opgegeven beoordelingstechnieken. Ten tweede zouden de testen op meer plaatsen kunnen uitgewerkt worden. Deze mogelijkheid werd beperkt door de verschillende corona-maatregelen die werden opgelegd door de overheid. Ten derde zou CoreML ook beter kunnen worden getest door middel van een proof-of-concept die in direct verband staat met de wayfinding-context, uit deze applicatie is het dan verder mogelijk om tests uit te voeren die veel doelgerichter zijn.

In de toekomst zou deze uitwerking meer in verband kunnen staan met de wayfinding context. Ten laatste kunnen de proeven grootschaliger worden getest, in plaats van 29 iteraties zou je dit kunnen uitbreiden naar 299. Deze uitwerking zou een beter beeld geven over de consistentie van elke AI-eigenschap.