%%=============================================================================
%% Inleiding
%%=============================================================================
\chapter{\IfLanguageName{dutch}{Inleiding}{Introduction}}
\label{ch:inleiding}

\section{\IfLanguageName{dutch}{Wayfinding}{Wayfinding}}
\label{sec:wayfinding}
Het begrip ''Wayfinding'' is op zich zeer breed, algemeen staat het gekend als het gedrag en denkvermogen dat nodig is om de weg terug te vinden in een specifieke omgeving. Om zich te verplaatsen heeft men reeds smartphones en GPS implementaties, wayfinding zit hier dus ook in verweven. In deze bachelorproef zal men bespreken hoe men wayfinding op een optimale manier kan toepassen met behulp van technische middelen.

\section{\IfLanguageName{dutch}{Probleemstelling}{Problem Statement}}
\label{sec:probleemstelling}

In The Pocket is een Belgisch IT-bedrijf dat zich focust op digitale producten, zij wensen een GPS-applicatie te implementeren dat wayfinding optimaliseert, dit betekent dat er geen fouten meer worden gemaakt bij het wijzen van de weg. Om deze toepassing te optimaliseren verlangt men gebruik te maken van AI en AR om omgevingsfactoren te detecteren, te analyseren, en bovendien de input te vertalen naar de AR omgeving. Als men bijvoorbeeld tegen een muur dreigt te lopen, dan kan AI dit corrigeren. In deze bachelorproef zal ik een onderzoek voeren dat resulteert in een overzicht van verschillende mogelijke algoritmes en/of aanpakken. Deze zal men kunnen toepassen bij het implementeren van de gewenste GPS-applicatie.

\section{\IfLanguageName{dutch}{Onderzoeksvraag}{Research question}}
\label{sec:onderzoeksvraag}

Wees zo concreet mogelijk bij het formuleren van je onderzoeksvraag. Een onderzoeksvraag is trouwens iets waar nog niemand op dit moment een antwoord heeft (voor zover je kan nagaan). Het opzoeken van bestaande informatie (bv. ``welke tools bestaan er voor deze toepassing?'') is dus geen onderzoeksvraag. Je kan de onderzoeksvraag verder specifiëren in deelvragen. Bv.~als je onderzoek gaat over performantiemetingen, dan 

Hoofdvraag: Welke bestaande technieken bestaan er reeds om aan de hand van AI en AR de drift in wayfinding te optimaliseren? Hoe kunnen we de wereld rondom de gebruiker herkennen, analyseren en bovendien de input op een bruikbare manier vertalen naar de AR omgeving?

Deelvraag: Wat is het optimale algoritme/techniek om de drift in wayfinding te optimaliseren aan de hand van AI en AR? 

\section{\IfLanguageName{dutch}{Onderzoeksdoelstelling}{Research objective}}
\label{sec:onderzoeksdoelstelling}

Wat is het beoogde resultaat van je bachelorproef? Wat zijn de criteria voor succes? Beschrijf die zo concreet mogelijk. Gaat het bv. om een proof-of-concept, een prototype, een verslag met aanbevelingen, een vergelijkende studie, enz.

De doelstelling van dit onderzoek is om een duidelijk overzicht te creëren van welke algoritmes en/of aanpakken goede prestaties zullen leveren bij het implementeren in de wayfinding context. Goede prestaties kan men vertalen in een applicatie (proof-of-concept) die zonder fouten, de juiste route zal aangeven.
De ultieme succesfactor van deze bachelorproef is het vinden van één of meerdere algoritmen die het bedrijf ''In The Pocket'' zou kunnen gebruiken bij het implementeren van de concrete GPS-applicatie.



\section{\IfLanguageName{dutch}{Opzet van deze bachelorproef}{Structure of this bachelor thesis}}
\label{sec:opzet-bachelorproef}

% Het is gebruikelijk aan het einde van de inleiding een overzicht te
% geven van de opbouw van de rest van de tekst. Deze sectie bevat al een aanzet
% die je kan aanvullen/aanpassen in functie van je eigen tekst.

De rest van deze bachelorproef is als volgt opgebouwd:

In Hoofdstuk~\ref{ch:stand-van-zaken} wordt een overzicht gegeven van de stand van zaken binnen het onderzoeksdomein, op basis van een literatuurstudie.

In Hoofdstuk~\ref{ch:methodologie} wordt de methodologie toegelicht en worden de gebruikte onderzoekstechnieken besproken om een antwoord te kunnen formuleren op de onderzoeksvragen.

% TODO: Vul hier aan voor je eigen hoofstukken, één of twee zinnen per hoofdstuk

In Hoofdstuk~\ref{ch:conclusie}, tenslotte, wordt de conclusie gegeven en een antwoord geformuleerd op de onderzoeksvragen. Daarbij wordt ook een aanzet gegeven voor toekomstig onderzoek binnen dit domein.