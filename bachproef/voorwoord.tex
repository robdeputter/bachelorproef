%%=============================================================================
%% Voorwoord
%%=============================================================================

\chapter*{\IfLanguageName{dutch}{Woord vooraf}{Preface}}
\label{ch:voorwoord}

%% TODO:
%% Het voorwoord is het enige deel van de bachelorproef waar je vanuit je
%% eigen standpunt (``ik-vorm'') mag schrijven. Je kan hier bv. motiveren
%% waarom jij het onderwerp wil bespreken.
%% Vergeet ook niet te bedanken wie je geholpen/gesteund/... heeft
Dit eindwerk vormt de afsluiter van mijn professionele bacheloropleiding Toegepaste Informatica, afstudeerrichting Mobiele Applicaties aan de HoGent. Ik kan terugkijken naar 3 fantastische jaren waar ik zeer veel heb bijgeleerd en veel nieuwe vrienden heb gemaakt.

Het onderzoek dat werd uitgevoerd in deze bachelorproef gaat vooral over Artificiële intelligentie, een domein binnen informatica dat in onze opleiding alleen op theoretisch vlak werd aangeboden. Om mijn kennis te verrijken en mij wat meer te verdiepen in welke mogelijkheden er zijn met deze technologie heb ik ervoor gekozen om dit te betrekken in het afsluitende opleidingsonderdeel. Ik ben een student die ernaar streeft om relevante technologieën te begrijpen en mogelijks ook te gebruiken in projecten. Alsook probeer ik zoveel mogelijk unieke ervaringen op te doen, in het laatste semester van mijn bachelor heb ik er dan ook voor gekozen om mijn studies in het buitenland af te ronden door middel van een buitenlandse stage. Dit aspect heeft zeker een effect gehad op het uitwerken van dit onderzoek, ik had veel minder resources ter beschikking, maar ze hebben me altijd geleerd om te roeien met de riemen die je hebt.
Uit het onderzoek dat ik heb uitgevoerd heb ik zeker bijgeleerd, ik ben veel meer op de hoogte van welke AI-eigenschappen een AI-framework te bieden heeft.

Het eindresultaat van deze bachelorproef zou nooit hetzelfde zijn zonder de hulp van mijn promotor Steven Van Impe en co-promotor Thijs Morlion, zij hebben mijn eindwerk zorgvuldig nagelezen en tips gegeven waar het nodig was. Ik wel hen beiden dan ook ten zeerste bedanken. Alsook wil ik mijn zus, Lise De Putter bedanken, zij heeft deze bachelorproef meerdere malen nagelezen om alle punten op de i te zetten.

\newpage
Veel appreciatie ben ik ook verschuldigd aan alle docenten en begeleiders van mijn opleiding Toegepaste Informatica, zij zijn elke dag in de weer om alle studenten op te leiden tot IT-professionals, wat geen makkelijke zaak is. Zij gaven mij ook meer toekomstperspectief en motivatie om zoveel mogelijk bij te leren.

Een speciaal woordje van dank gaat uit naar mijn dierbare vrienden en studiegenoten die mij altijd hebben gesteund tijdens moeilijke (en gewone) tijden. Zonder hen zou deze opleiding nooit hetzelfde zijn geweest.

Ik wens u veel leesplezier toe.

Rob De Putter

Kerksken, 25 mei 2020


