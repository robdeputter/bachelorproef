%%=============================================================================
%% Samenvatting
%%=============================================================================

% TODO: De "abstract" of samenvatting is een kernachtige (~ 1 blz. voor een
% thesis) synthese van het document.
%
% Deze aspecten moeten zeker aan bod komen:
% - Context: waarom is dit werk belangrijk?
% - Nood: waarom moest dit onderzocht worden?
% - Taak: wat heb je precies gedaan?
% - Object: wat staat in dit document geschreven?
% - Resultaat: wat was het resultaat?
% - Conclusie: wat is/zijn de belangrijkste conclusie(s)?
% - Perspectief: blijven er nog vragen open die in de toekomst nog kunnen
%    onderzocht worden? Wat is een mogelijk vervolg voor jouw onderzoek?
%
% LET OP! Een samenvatting is GEEN voorwoord!

%%---------- Nederlandse samenvatting -----------------------------------------
%
% TODO: Als je je bachelorproef in het Engels schrijft, moet je eerst een
% Nederlandse samenvatting invoegen. Haal daarvoor onderstaande code uit
% commentaar.
% Wie zijn bachelorproef in het Nederlands schrijft, kan dit negeren, de inhoud
% wordt niet in het document ingevoegd.

\IfLanguageName{english}{%
\selectlanguage{dutch}
\chapter*{Samenvatting}
\lipsum[1-4]
\selectlanguage{english}
}{}

%%---------- Samenvatting -----------------------------------------------------
% De samenvatting in de hoofdtaal van het document

\chapter*{\IfLanguageName{dutch}{Samenvatting}{Abstract}}

Artificiële Intelligentie (AI) en Augmented Reality (AR) zijn bijna niet meer weg te denken binnen de IT-wereld. De laatste jaren kenden beide technologieën een groei dankzij hun vooruitgang, zo werden deze technologieën accurater en performanter. AR werd geïntroduceerd op sociale media door middel van Snapchat -en Instagramfilters, ook Pokemon Go maakt gebruik van deze technologie. AI bestaat reeds langer, één van de bekendste hulpmiddelen die gebruik maken van deze technologie zijn Google Home, Alexa en Siri. 

In deze bachelorproef zullen deze twee technologieën gecombineerd worden om een indoor wayfinding applicatie te optimaliseren. Wayfinding kan je verstaan als het denkvermogen dat nodig is om de weg terug te vinden in een specifieke omgeving. Deze applicatie zal de gebruiker de weg wijzen met behulp van 3D-pijlen die op het mobiele toestel zullen verschijnen, hier wordt AR geïntroduceerd in het verhaal. AR zal ervoor zorgen dat de 3D-pijlen op een gepaste manier worden getoond. Het bepalen hoe de pijlen worden getoond op het scherm is van cruciaal belang, het is zeer belangrijk dat deze zich bijvoorbeeld niet door muren begeven. Daarom is het belangrijk dat de omgeving op een zo goed mogelijke manier in kaart wordt gebracht. Het in kaart brengen van de omgeving is een aspect dat reeds wordt gedaan door AR, maar dit kan geoptimaliseerd worden door AI. Artificiële intelligentie zal bijvoorbeeld de muren van het grondoppervlak kunnen onderscheiden, waardoor dergelijke fouten niet meer zullen voorkomen.

In dit onderzoek werd bestudeerd welke mogelijke AI-frameworks in staat zijn om dergelijke zaken uit te voeren, vervolgens werden een paar van deze frameworks getest en geëvalueerd op basis van beoordelingstechnieken die de verschillende noden van een wayfinding applicatie nastreven. Deze frameworks werden tegenover elkaar gezet en diegene met de beste resultaten werd verkozen als 'eindwinnaar'.

Door de beperkte middelen tijdens het uitvoeren van dit onderzoek kan deze bachelorproef zeker nog worden geoptimaliseerd. In het vervolg op deze bachelorproef kunnen proeven op een grootschaligere manier worden uitgevoerd. Het framework met de beste resultaten zou ook in een praktisch voorbeeld kunnen worden uitgewerkt.

